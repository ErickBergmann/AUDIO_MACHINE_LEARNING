%\documentclass[10pt,a4paper]{report}
\documentclass{exam}
%encoding
%--------------------------------------
\usepackage[T1]{fontenc}
\usepackage[utf8]{inputenc}
%--------------------------------------
 
%Portuguese-specific commands
%--------------------------------------
\usepackage[portuguese]{babel}
%---------------------------------
\usepackage{graphicx}
\usepackage{comment}
\usepackage{exercise}
\usepackage{enumerate}
\usepackage{amsmath,amssymb,amsthm}
\usepackage[shortlabels]{enumitem}
\usepackage{algorithmic}
\usepackage[ruled,vlined]{algorithm2e}

\usepackage{tikz}
\usepackage{tikz-3dplot}
\usepackage{graphicx}
\usetikzlibrary{%
    decorations.pathreplacing,%
    decorations.pathmorphing%
}
\usetikzlibrary{calc}
\usetikzlibrary{positioning}
%%%%%%%%%%%%%%%%%%%%%%%%%%%%%%%%%%%%%%%%%%%%%%%%%%%%%%%%%%%%%%%%%%%
\newtheorem{theorem}{Theorem}
\newtheorem*{theorem*}{Theorem}
\newtheorem{conjecture}{Conjecture}
\newtheorem{lemma}{Lemma}
\newtheorem{corollary}{Corollary}
\newtheorem{proposition}{Proposition}
\newtheorem{definition}{Definition}
\theoremstyle{definition}
\newtheorem{example}{Example}
\newtheorem{remark}{Observações}

\newcommand\x{\times}
\newcommand\bigzero{\makebox(0,0){\text{\huge0}}}
\newcommand*{\bord}{\multicolumn{1}{c|}{}}
%%%%%%%%%%%%%%%%%%%%%%%%%%%%%%%%%%%%%%%%%%%%%%%%%%%%%%%%%%%%%%%%%%%

%\usepackage[dvipsnames]{xcolor}
%\usepackage{hyperref}

%%%%%%%%%%%%%%%%%%%%%%%%%%%%%%%%%%%%%%%%%%%%%%%%%%%%%%%%%%%%%%%%%%%
% Otra opcion de colores:
%\newcommand\myshade{85}
%\colorlet{mylinkcolor}{violet}
%\colorlet{mycitecolor}{YellowOrange}
%\colorlet{myurlcolor}{Aquamarine}
%\usepackage{hyperref}
%\hypersetup{
%  linkcolor  = mylinkcolor!\myshade!black,
%  citecolor  = mycitecolor!\myshade!black,
%  urlcolor   = myurlcolor!\myshade!black,
%  colorlinks = true,
%}

\usepackage{hyperref}
\usepackage{cleveref}
\hypersetup{
  linkcolor  = blue,
  citecolor  = blue,
  urlcolor   = blue,
  colorlinks = true,
}
%%%%%%%%%%%%%%%%%%%%%%%%%%%%%%%%%%%%%%%%%%%%%%%%%%%%%%%%%%%%%%%%%%%

\usepackage{mathtools}
\usepackage[numbered,framed]{matlab-prettifier}
\usepackage{filecontents}

\newcommand{\vetx}{\mathbf{x}}
%\newcommand{\cos}{\text{cos}}

\DeclarePairedDelimiter\abs{\lvert}{\rvert}%
\DeclarePairedDelimiter\norm{\lVert}{\rVert}%

\usepackage{listings}
\usepackage{color} %red, green, blue, yellow, cyan, magenta, black, white
\definecolor{mygreen}{RGB}{28,172,0} % color values Red, Green, Blue
\definecolor{mylilas}{RGB}{170,55,241}

\lstset{
    language=Octave, %% Troque para PHP, C, Java, etc... bash é o padrão
    breaklines=true,%
    morekeywords={matlab2tikz},
    keywordstyle=\color{blue},%
    morekeywords=[2]{1}, keywordstyle=[2]{\color{black}},
    identifierstyle=\color{black},%
    stringstyle=\color{mylilas},
    commentstyle=\color{mygreen},%
    showstringspaces=false,%without this 
    numbers=left,
    backgroundcolor=\color{gray!10},
    frame=single,
    tabsize=2,
    rulecolor=\color{black!30},
    title=\lstname,
    escapeinside={\%*}{*)},
    breaklines=true,
    breakatwhitespace=true,
    framextopmargin=2pt,
    framexbottommargin=2pt,
    extendedchars=false,
    inputencoding=utf8
}

%%%%%%%%%%%%%%%%%%%%%%%%
%%%%%%%%%%%%%%%%%%%%%%%%
%%%%%%%%%%%%%%%%%%%%%%%%
% Default fixed font does not support bold face
\DeclareFixedFont{\ttb}{T1}{txtt}{bx}{n}{12} % for bold
\DeclareFixedFont{\ttm}{T1}{txtt}{m}{n}{12}  % for normal

% Custom colors
\usepackage{color}
\definecolor{deepblue}{rgb}{0,0,0.5}
\definecolor{deepred}{rgb}{0.6,0,0}
\definecolor{deepgreen}{rgb}{0,0.5,0}

\usepackage{listings}

% Python style for highlighting
\newcommand\pythonstyle{\lstset{
language=Python,
basicstyle=\ttm,
otherkeywords={self},             % Add keywords here
keywordstyle=\ttb\color{deepblue},
emph={MyClass,__init__},          % Custom highlighting
emphstyle=\ttb\color{deepred},    % Custom highlighting style
stringstyle=\color{deepgreen},
frame=tb,                         % Any extra options here
showstringspaces=false            % 
}}


% Python environment
\lstnewenvironment{python}[1][]
{
\pythonstyle
\lstset{#1}
}
{}

% Python for external files
\newcommand\pythonexternal[2][]{{
\pythonstyle
\lstinputlisting[#1]{#2}}}

% Python for inline
\newcommand\pythoninline[1]{{\pythonstyle\lstinline!#1!}}
%%%%%%%%%%%%%%%%%%%%%%%%
%%%%%%%%%%%%%%%%%%%%%%%%
%%%%%%%%%%%%%%%%%%%%%%%%

\oddsidemargin = 8pt
\title{Mini Quiz 2: Conceptos Básicos y Notación.}
\author{Profesor: Rodolfo Anibal Lobo}
\date{Septiembre 2023}

\begin{document}
\maketitle

%\begin{table}[h!]
%\centering
%\begin{tabular}{|l|l|l|l|l|l|l|}
%\hline
% Pregunta & $(1)$ & $(2)$ & $(3)$ & $(4)$  & $(5)$ & Total \\ \hline
% Q1 & $\times$  &$\times$   & $\times$   & $\times$ & $\times$ & $\mathtt{5pts}$  \\ \hline
% Q2 & $\times$ & $\times$  & $\times$   & $\times$   & $\times$  & $\mathtt{5pts}$  \\ \hline
%\end{tabular}
%\end{table}


\section*{Instrucciones}
\thispagestyle{empty}
Lea atentamente las preguntas y conteste de manera breve y clara sus respuestas. Recuerde que en negrita expresamos vectores $\mathbf{x}$ fila o coulmna, en cursiva y con sub-indices $x_i$ componentes o escalares, donde $i\in \mathbb{Z}^{+}$, en algunos casos letras mayúsculas pueden representar indistintamente vectores o matrices. \textbf{Elija 3 preguntas}.

\begin{enumerate}
\item ¿Qué es una regresión lineal?, defina de la manera más formal posible y de un ejemplo de aplicación.
\item Escriba a un costado de cada $\mathtt{print}$ lo que imprimirá la función en cada caso:
    \begin{python}
    import numpy as np
     x = [1,2,3]
     y = [4,5,6]
     print(x+y)
    
     x_vec = np.array([1,2,3])
     y_vec = np.array([4,5,6])
     print(x_vec+y_vec)
 
    \end{python}
\item Supongamos que tenemos un conjunto de audios $A = \{\mathbf{a}_1, \mathbf{a}_2,\dots,\mathbf{a}_n\}$ donde cada $\mathbf{a}_i$ con $i = 1,\dots, n$ es un vector con $m$ muestras, explícitamente $\mathbf{a}_i = [x_0,x_1,x_2,\dots,x_m]$. Si queremos entrenar un modelo para hallar los parámetros $\theta$ en la ecuación:
\begin{equation}
    D \theta = Y + \epsilon
\end{equation}
donde $\epsilon$ es una matriz de error, $D$ es una matriz que tiene por filas las primeras $p<m$ muestras de un audio e $Y$ una matriz con las últimas $m-p$ muestras. Entonces, ¿este es un problema de regresión o clasificación?, ¿por qué?

\item Dadas las siguientes ecuaciones normales:
    \begin{equation}
      \Theta = (X^{T}X)^{-1}X^{T}\mathbf{y}
    \end{equation}
    Si $\mathbf{x} \in \mathbb{R}^{1\times N}$ es una fila de la matriz $X$, es decir $\mathbf{x}=[x_1,x_2,\dots,x_N]$ nuestros datos, para los cuales tenemos 250 ejemplos. Además sea $\Theta\in \mathbb{R}^{6\times 1}$ e $y_i \in \mathbb{R}$. ¿Cuáles son las dimensiones de las siguientes matrices-vectores?
    \begin{enumerate}
        \item $X^{T}$
        \item $X^{T}X$
        \item $(X^{T}X)^{-1}$
        \item $X^{T}\mathbf{y}$
        \item $\mathbf{x}$
        \item $(X^{T}X)^{-1}X^{T}$
        \item $\mathbf{y}$
    \end{enumerate}
\end{enumerate}
\thispagestyle{empty}
\end{document}
