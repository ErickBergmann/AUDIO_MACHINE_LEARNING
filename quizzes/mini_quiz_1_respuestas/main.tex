%\documentclass[10pt,a4paper]{report}
\documentclass{exam}
%encoding
%--------------------------------------
\usepackage[T1]{fontenc}
\usepackage[utf8]{inputenc}
%--------------------------------------
 
%Portuguese-specific commands
%--------------------------------------
\usepackage[portuguese]{babel}
%---------------------------------
\usepackage{graphicx}
\usepackage{comment}
\usepackage{exercise}
\usepackage{enumerate}
\usepackage{amsmath,amssymb,amsthm}
\usepackage[shortlabels]{enumitem}
\usepackage{algorithmic}
\usepackage[ruled,vlined]{algorithm2e}

\usepackage{tikz}
\usepackage{tikz-3dplot}
\usepackage{graphicx}
\usetikzlibrary{%
    decorations.pathreplacing,%
    decorations.pathmorphing%
}
\usetikzlibrary{calc}
\usetikzlibrary{positioning}
%%%%%%%%%%%%%%%%%%%%%%%%%%%%%%%%%%%%%%%%%%%%%%%%%%%%%%%%%%%%%%%%%%%
\newtheorem{theorem}{Theorem}
\newtheorem*{theorem*}{Theorem}
\newtheorem{conjecture}{Conjecture}
\newtheorem{lemma}{Lemma}
\newtheorem{corollary}{Corollary}
\newtheorem{proposition}{Proposition}
\newtheorem{definition}{Definition}
\theoremstyle{definition}
\newtheorem{example}{Example}
\newtheorem{remark}{Observações}

\newcommand\x{\times}
\newcommand\bigzero{\makebox(0,0){\text{\huge0}}}
\newcommand*{\bord}{\multicolumn{1}{c|}{}}
%%%%%%%%%%%%%%%%%%%%%%%%%%%%%%%%%%%%%%%%%%%%%%%%%%%%%%%%%%%%%%%%%%%

%\usepackage[dvipsnames]{xcolor}
%\usepackage{hyperref}

%%%%%%%%%%%%%%%%%%%%%%%%%%%%%%%%%%%%%%%%%%%%%%%%%%%%%%%%%%%%%%%%%%%
% Otra opcion de colores:
%\newcommand\myshade{85}
%\colorlet{mylinkcolor}{violet}
%\colorlet{mycitecolor}{YellowOrange}
%\colorlet{myurlcolor}{Aquamarine}
%\usepackage{hyperref}
%\hypersetup{
%  linkcolor  = mylinkcolor!\myshade!black,
%  citecolor  = mycitecolor!\myshade!black,
%  urlcolor   = myurlcolor!\myshade!black,
%  colorlinks = true,
%}

\usepackage{hyperref}
\usepackage{cleveref}
\hypersetup{
  linkcolor  = blue,
  citecolor  = blue,
  urlcolor   = blue,
  colorlinks = true,
}
%%%%%%%%%%%%%%%%%%%%%%%%%%%%%%%%%%%%%%%%%%%%%%%%%%%%%%%%%%%%%%%%%%%

\usepackage{mathtools}
\usepackage[numbered,framed]{matlab-prettifier}
\usepackage{filecontents}

\newcommand{\vetx}{\mathbf{x}}
%\newcommand{\cos}{\text{cos}}

\DeclarePairedDelimiter\abs{\lvert}{\rvert}%
\DeclarePairedDelimiter\norm{\lVert}{\rVert}%

\usepackage{listings}
\usepackage{color} %red, green, blue, yellow, cyan, magenta, black, white
\definecolor{mygreen}{RGB}{28,172,0} % color values Red, Green, Blue
\definecolor{mylilas}{RGB}{170,55,241}

\lstset{
    language=Octave, %% Troque para PHP, C, Java, etc... bash é o padrão
    breaklines=true,%
    morekeywords={matlab2tikz},
    keywordstyle=\color{blue},%
    morekeywords=[2]{1}, keywordstyle=[2]{\color{black}},
    identifierstyle=\color{black},%
    stringstyle=\color{mylilas},
    commentstyle=\color{mygreen},%
    showstringspaces=false,%without this 
    numbers=left,
    backgroundcolor=\color{gray!10},
    frame=single,
    tabsize=2,
    rulecolor=\color{black!30},
    title=\lstname,
    escapeinside={\%*}{*)},
    breaklines=true,
    breakatwhitespace=true,
    framextopmargin=2pt,
    framexbottommargin=2pt,
    extendedchars=false,
    inputencoding=utf8
}

%%%%%%%%%%%%%%%%%%%%%%%%
%%%%%%%%%%%%%%%%%%%%%%%%
%%%%%%%%%%%%%%%%%%%%%%%%
% Default fixed font does not support bold face
\DeclareFixedFont{\ttb}{T1}{txtt}{bx}{n}{12} % for bold
\DeclareFixedFont{\ttm}{T1}{txtt}{m}{n}{12}  % for normal

% Custom colors
\usepackage{color}
\definecolor{deepblue}{rgb}{0,0,0.5}
\definecolor{deepred}{rgb}{0.6,0,0}
\definecolor{deepgreen}{rgb}{0,0.5,0}

\usepackage{listings}

% Python style for highlighting
\newcommand\pythonstyle{\lstset{
language=Python,
basicstyle=\ttm,
otherkeywords={self},             % Add keywords here
keywordstyle=\ttb\color{deepblue},
emph={MyClass,__init__},          % Custom highlighting
emphstyle=\ttb\color{deepred},    % Custom highlighting style
stringstyle=\color{deepgreen},
frame=tb,                         % Any extra options here
showstringspaces=false            % 
}}


% Python environment
\lstnewenvironment{python}[1][]
{
\pythonstyle
\lstset{#1}
}
{}

% Python for external files
\newcommand\pythonexternal[2][]{{
\pythonstyle
\lstinputlisting[#1]{#2}}}

% Python for inline
\newcommand\pythoninline[1]{{\pythonstyle\lstinline!#1!}}
%%%%%%%%%%%%%%%%%%%%%%%%
%%%%%%%%%%%%%%%%%%%%%%%%
%%%%%%%%%%%%%%%%%%%%%%%%

\oddsidemargin = 8pt
\title{Mini Quiz 1: Historia y primeras definiciones.}
\author{Profesor: Rodolfo Anibal Lobo}
\date{Septiembre 2023}

\begin{document}
\maketitle

%\begin{table}[h!]
%\centering
%\begin{tabular}{|l|l|l|l|l|l|l|}
%\hline
% Pregunta & $(1)$ & $(2)$ & $(3)$ & $(4)$  & $(5)$ & Total \\ \hline
% Q1 & $\times$  &$\times$   & $\times$   & $\times$ & $\times$ & $\mathtt{5pts}$  \\ \hline
% Q2 & $\times$ & $\times$  & $\times$   & $\times$   & $\times$  & $\mathtt{5pts}$  \\ \hline
%\end{tabular}
%\end{table}


\section*{Instrucciones}
\thispagestyle{empty}
Lea atentamente las preguntas y conteste de manera breve y clara sus respuestas. Se utilizarán los primeros 10 minutos de la clase como máximo para realizar esta tarea. \textbf{Debe elegir una pregunta y contestarla.}

\begin{enumerate}
    \item ¿Cuál es la definición de Machine Learning dada por Arthur Samuel (1950)?. Luego, mencione y expliqué en términos generales las otras dos descripciones vistas en clases.
    \begin{enumerate}
    \item \textcolor{cyan}{The field of study that gives computers the ability to learn without explicitly being programmed. El campo de estudio que da a las computadoras la habilidad de aprender sin ser explícitamente programadas.} \textcolor{black}{}
    \end{enumerate}
    \item Defina aprendizaje supervisado y no supervisado. De un ejemplo de uso en cada caso. 
    \begin{enumerate}
    \item \textcolor{cyan}{\textbf{Aprendizaje Supervisado}: Dado un conjunto de datos de entrada con sus correspondientes etiquetas, que podemos considerar como respuestas correctas, queremos entrenar un modelo capaz de predecir tales etiquetas y que tenga la capacidad de generalizar sus respuestas ante nuevas entradas. \textit{Ejemplos de uso}: clasificación, regresión, traducción, detector de spam, clasificación de sonidos, clasificación de emociones etc.}\textcolor{black}{}
    \item \textcolor{cyan}{\textbf{Aprendizaje no supervisado}: a diferencia del aprendizaje supervisado, el aprendizaje no supervisado no utiliza etiquetas o respuestas correctas y solo basado en la estructura de los datos permite agruparlos en "clusters" o nubes de datos similares, también podríamos decir "estructuras" similares. \textit{Ejemplos de uso}: reducción de dimensión, clustering y asociación. }\textcolor{black}{}
    \end{enumerate}
    \item ¿Cuando trabajamos en $\mathtt{python}$, qué tipos de estructuras de datos podemos encontrar? (Nombre dos y sus características. Si no recuerda alguna característica en particular y recuerda el código, escriba un código simple y describa que es lo que hace y el resultado que entregaría).
    \begin{enumerate}
    \item \textcolor{cyan}{\textbf{Listas}: lista de objetos, pueden ser números, strings, funciones etc. Puede mezclar tipo de objetos. Podemos acceder a estos objetos a través de indices enteros.} 
    \begin{python}
        # lista vacia (la palabra "list" es reservada)
        lista = []
        lista.append(1)
        lista.append("hola")
        print(lista[0])
        1
        print(lista[1])
        "hola"
    \end{python}
    \textcolor{cyan}{
    tienen asociados varios métodos más a parte de $\mathtt{append}$ para más detalles revisar \href{https://docs.python.org/3/tutorial/datastructures.html}{LINK}. }\textcolor{black}{}
    \item \textcolor{cyan}{\textbf{Diccionarios}: Consisten en una asociación llave-valor que permite almacenar, eliminar información respecto a una llave, tal y como una pequeña "memoria". Las llaves son valores únicos e inmutables. Para más información sobre mutabilidad} \href{https://realpython.com/python-mutable-vs-immutable-types/#:~:text=Python%20has%20both%20mutable%20and,you%20as%20a%20Python%20developer.}{LINK}
    \item \textcolor{cyan}{\textbf{Strings}: Secuencias de caracteres individuales. Cada caracter es indexado. Es posible aplicar diferentes métodos a strings como por ejemplo} $\mathtt{.lower()}$ \textcolor{cyan}{que deja todas las letras en minúscula (lowercase).}
    \item \textcolor{cyan}{\textbf{Vectores}: A través de la librería $\mathtt{numpy}$ es posible crear vectores que se rigen bajo las leyes del álgebra lineal. Estas estructuras pueden operarse (suma, resta, producto punto etc). }
    \item \textcolor{cyan}{\textbf{Conjuntos}: Al igual que en teoría de conjuntos, puedes unir, intersectar y aplicar operaciones de conjuntos en estos objetos que simulan un conjunto clásico.} 
    \end{enumerate}
\end{enumerate}
\end{document}
